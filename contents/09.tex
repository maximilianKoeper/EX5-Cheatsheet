\section{Halbleiter}

\subsubsection*{Eigenleitung}
\begin{equation*}
    \begin{aligned}
        \sigma &= e (n \mu_n + p \mu_p) \\
        \mu &= \frac{e \tau}{m^*} \\
        n &= \int_{E_L}^\infty D_L(E) f(E,T) dE \\
        p &= \int_{-\infty}^{E_V} D_V(E) (1-f(E,T)) dE
    \end{aligned}
\end{equation*}

\subsubsection*{intrinsische Ladungsträgerdichte}
\begin{equation*}
    \begin{aligned}
        D_L(E) &= \frac{1}{2 \pi^2} \left(\frac{2 m_n^*}{\hbar^2}\right)^{\frac{3}{2}} \sqrt{E-E_L} \\
        D_V(E) &= \frac{1}{2 \pi^2} \left(\frac{2 m_p^*}{\hbar^2}\right)^{\frac{3}{2}} \sqrt{E_V-E} \\
        & \Rightarrow \text{Freie Ladungen:}\\
        n &= N_L \exp\left(-\frac{E_L-E_F}{k_B T}\right) \\
        p &= N_V \exp\left(\frac{E_V - E_F}{k_B T}\right)
    \end{aligned}
\end{equation*}

\subsubsection*{Lage des Fermi-Niveaus}
\begin{equation*}
    \begin{aligned}
        n_i &= p_i = \sqrt{N_L N_V} \exp \left(- \frac{E_g}{2 k_B T}\right) \\
        E_F &= \frac{E_L + E_V}{2} + \frac{k_B T}{2} \ln \left(\frac{N_V}{N_L}\right) \\
            &= \frac{E_L + E_V}{2} + \frac{3 k_B T}{4} \ln \left(\frac{m_p^*}{m_n^*}\right)
    \end{aligned}
\end{equation*}

\subsubsection*{Ladungsträgerdichte \& Fermi-Niveau}
Wahrscheinlichket, dass eine Störstelle nicht ionisiert ist:
\begin{equation*}
    \begin{aligned}
        \frac{n_D^0}{n_D} &= 2\frac{1}{e^{(E_D - E_F)/k_B T} +1} \\
        \frac{n_A^0}{n_A} &= 4\frac{1}{e^{(E_F - E_A)/k_B T} +1} \\
    \end{aligned}
\end{equation*}
für $n_D \gg n_A$: 
\begin{equation*}
    \begin{aligned}
        \frac{n(n_A + n)}{n_D - n_A - n} &= N_L \exp \left(- \frac{E_d}{k_B T}\right) \\
        E_d & = E_L - E_D
    \end{aligned}
\end{equation*}

\textbf{- Sehr tiefe Temperaturen:}\\
    Kompensationsbereich: \\
    $n \approxeq \frac{n_D n }{n_A} e^{-E_d / k_B T}$ \\
    $ E_F \approxeq E_L - E_d + k_B T \ln \left(\frac{n_D}{n_A}\right) $ \\
    $E_F$ wird durch Donatoren bestimmt. \\
\textbf{- Tiefe Temperaturen:} \\
    Störstellenreserve: \\
    $n \approx \sqrt{N_L n_D} \exp\left(- \frac{E_d}{2 k_B T}\right)$ \\
    $E_F \approx E_L - \frac{E_d}{2} + \frac{k_B T}{2} \ln \left(\frac{N_L}{n_D}\right)$ \\
    $E_F$ liegt etwa in der Mitte der Bänder.\\
\textbf{- Mittlere Temperaturen:} \\
    Störstellenerschöpfung: \\
    $n \approx n_D$ \\
    $E_F \approx E_L - k_B T \ln \left(\frac{N_L}{n_D}\right)$ \\
    Die Temperatur ist hoch genug, um alle Störstellen zu ionisieren, aber noch zu klein, um eine große Zahl von Ladungsträgern aus dem
    Valenz- ins Leitungsband anzuregen.\\
\textbf{- Hohe Temperaturen:} \\
    Eigenleitung: \\
    $n^2 = N_L^2 \exp \left(- \frac{E_g}{k_B T}\right)$ \\
    $E_F = \frac{E_L + E_V}{2}$ \\

\subsubsection*{p-n-Übergang:}
\begin{equation*}
    \begin{aligned}
        \abs{j^f} &= \abs{j^d} = a(T) \exp \left(- \frac{eV_d}{k_B T}\right) \\
        V_D &:= \text{Diffusionsspannung}
    \end{aligned}
\end{equation*}

\subsubsection*{Shockley-Gleichung:}
\begin{equation*}
    \begin{aligned}
        j(U) &= j_S \left(e^{eU / k_B T} -1\right)
    \end{aligned}
\end{equation*}