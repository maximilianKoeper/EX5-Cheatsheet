\section{Gitterdynamik}

% ###############################

\subsection*{Lineare Kette}

\begin{equation*}
    \begin{aligned}
        F_s &= \sum_n C_n \left(u_{s+n} - u_s\right) = M \frac{d^2 u_s}{dt^2} \\
        u_{s+n} &= U \exp \left(-i\left(\omega t - q (s+n)a\right)\right)
    \end{aligned}
\end{equation*}

\subsubsection*{Nächste Nachbar Wechselwirkung:}
\begin{equation*}
    \begin{aligned}
        M \frac{d^2 u_s}{dt^2} &= \sum_{n>0} C \left(u_{s+1} -2u_s + u_{s-1}\right) \\
        u_{s \pm 1} &= U \exp \left(-i \left(\omega t - q (s \pm 1)a\right)\right) \\
        \omega &= 2 \sqrt{\frac{C}{M}} \abs{\sin \left(\frac{qa}{2}\right)}
    \end{aligned}
\end{equation*}

\subsubsection*{$\Gamma$-Punkt (Schall)}
\begin{equation*}
    \begin{aligned}
        \lambda &\gg a: qa \rightarrow 0 \; \sin(x) \approx x: \\
        \omega &= \sqrt{\frac{a^2C}{M}}q = v q \\
        v_p &= \frac{\omega}{q} \approx \frac{\partial \omega}{\partial q} = v_g \\
        v_{Schall} &= \left.\frac{\partial \omega}{\partial q} \right|_{q \rightarrow 0}
    \end{aligned}
\end{equation*}

\subsubsection*{Grenzfläche der 1. BZ. $q = \pm \frac{\pi}{a}$}
\begin{equation*}
    \nu_gr = \frac{\partial \omega}{\partial q} = 0
\end{equation*}
Benachbarte Gitterebenen schwingen gegenphasig:
\begin{equation*}
    \frac{u_{s+1}}{u_s} = \exp \left(\pm i \pi\right) = -1
\end{equation*}

\subsection*{Zweiatomige Basis}

\begin{equation*}
    \begin{aligned}
        \omega_{a,o}^2 &= \frac{C}{\mu} \mp C \sqrt{\frac{1}{\mu^2}- \frac{4}{M_1M_2} \sin^2\left(\frac{qa}{2}\right)} \\
        \omega_{a,o}^2 &= \frac{C'}{M} \mp \frac{1}{M} \sqrt{(C')^2- 4C_1 C_2 \sin^2\left(\frac{qa}{2}\right)} \\
        \frac{1}{\mu} &= \frac{1}{M_1} + \frac{1}{M_2}; \quad C' = C_1 + C_2
    \end{aligned}
\end{equation*}

\subsubsection*{Grenzfälle}
$\Gamma-$Punkt:
\begin{equation*}
    \begin{aligned}
        \omega_{ak} & \approx 0 \\
        \omega_{op}^2 &= \frac{2C}{\mu} = const.
    \end{aligned}
\end{equation*}
Grenze der 1. BZ.:
\begin{equation*}
    \begin{aligned}
        \omega_{ak}^2 &= \frac{2C}{M_1} \\
        \omega_{op}^2 &= \frac{2C}{M_2}
    \end{aligned}
\end{equation*}

\subsubsection*{p-atomige Basis}

\begin{itemize}
    \itemsep 0pt
    \item 3 akustische Zweige (1xL, 2xT)
    \item 3(p-1) optische Zweige
\end{itemize}

\subsubsection*{Übernächste NN:}
\begin{equation*}
    \begin{aligned}
        \omega^2(q) &= \frac{4C}{M} \sin^2\left(\frac{qa}{2}\right) \left[1+\frac{4}{\nu}\cos^2\left(\frac{qa}{2}\right)\right]
    \end{aligned}
\end{equation*}
