\section{Strukturbestimmung}

% ###############################
\subsection*{Reziprokes-Gitter}
\begin{equation*}
    \begin{aligned}
        a_i \cdot b_i &= 2 \pi \delta_{ij} = \begin{cases}
            2 \pi & \, \text{für } i = j \\
            0 & \, \text{sonst}
            \end{cases}
    \end{aligned}
\end{equation*}
\begin{equation*}
    \begin{aligned}
        b_1 &= \frac{2 \pi}{V_Z} (a_2 \times a_3) \\
        b_2 &= \frac{2 \pi}{V_Z} (a_3 \times a_1) \\
        b_3 &= \frac{2 \pi}{V_Z} (a_1 \times a_2) \\
    \end{aligned}
\end{equation*}

% ###############################
\subsection*{Bragg-Bedingung}

\begin{equation*}
    \begin{aligned}
        \lambda &= 2d_{hkl} \sin(\Theta) \; ; \; \lambda n = 2d \sin(\Theta)
    \end{aligned}
\end{equation*}

% ###############################
\subsection*{Strukturfaktor}

\begin{equation*}
    \begin{aligned}
        r_{\alpha} &= u_\alpha a_1 + v_\alpha a_2 + \omega_\alpha a_3 \\ 
        S_{hkl} &= \sum_\alpha f_\alpha (G) e^{-2\pi i (hu_\alpha + kv_\alpha + l\omega_\alpha)}
    \end{aligned}
\end{equation*}
 
\begin{equation*}
    \begin{aligned}
        &\text{kubisch primitv, zweiatomige Basis CsCl} \\
        S_{hkl} &= \begin{cases}
            f_{Cs} + f_{Cl} \pi & \, h+k+l \text{ gerade} \\
            f_{Cs} - f_{Cl} & \, h+k+l \text{ ungerade}
            \end{cases} \\
        &\text{kubisch raumzentriert, einfache Basis} \\
        S_{hkl} &= \begin{cases}
            2f \pi & \, h+k+l \text{ gerade} \\
            0 & \, h+k+l \text{ ungerade}
            \end{cases} \\
        &\text{kubisch flächenzentriert, einfache Basis} \\
        S_{hkl} &= \begin{cases}
            4f \pi & \, \text{ alle gerade / ungerade} \\
            0 & \, sosnt.
            \end{cases}
    \end{aligned}
\end{equation*}

% ###############################
\subsection*{Atom-Strukturfaktor}

\begin{equation*}
    \begin{aligned}
        f_\alpha  (K) &= \int_{V_\alpha} \rho_{\alpha} (r) e^{i K \cdot r} dV \\
        & \text{Neutronen-Streuung } K\cdot r \ll 1: \\
        f_{\alpha} &\approx \int_0^{R_\alpha} 4 \pi r^2 \rho(r) dr = Z \\
        & \text{Röntgen-Streuung:} \\
        \rho(r) &= \abs{\Psi_0(r)}^2 = \frac{1}{\pi a_0^3} e^{-2r/a_0} \\
        f_H(K) &= \frac{1}{\left[1+ \left(\frac{1}{2}a_0K\right)^2\right]^2}
    \end{aligned}
\end{equation*}