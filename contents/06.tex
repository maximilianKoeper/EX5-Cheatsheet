\section{Thermische Eigenschaften}

\subsubsection*{Wärmekapazität}

\begin{equation*}
    \begin{aligned}
        C_V &= \left(\frac{\partial U}{\partial T}\right)_V \\
        U &= 6N \frac{1}{2} k_B T = 3 N k_B T \\
        C_V &= 3Nk_B = 3R \; \text{ Dulong-Petit}
    \end{aligned}
\end{equation*}

\begin{equation*}
    U = \int_0^\infty E D(E) \langle n(E,T) \rangle dE
\end{equation*}

Zustandsdichte im q-Raum:
\begin{equation*}
    \begin{aligned}
        V_q &= \left(\frac{2 \pi}{L}\right)^3 \rightarrow \rho_q = \frac{1}{V_q} = \frac{V}{(2\pi)^3} \\
    \end{aligned}
\end{equation*}
Zahl der Zustände:
\begin{equation*}
    \begin{aligned}
        D(q) dq &= \rho_q 4 \pi q^2 dq = \frac{V}{2 \pi} q^2 dq \\
        q & \rightarrow \omega \text{(für isotrope Kristalle)} \\
        D(\omega) d\omega &= D(q) \frac{dq}{d\omega} d\omega = \frac{V}{2 \pi^2} q^2 \frac{dq}{d \omega} d\omega
    \end{aligned}
\end{equation*}

\subsubsection*{--> Debye-Näherung}
Lineare Dispersion $\omega = \nu q$

\begin{equation*}
    \begin{aligned}
        D(\omega)d \omega &= \frac{V}{2 \pi^2} \frac{\omega^2}{\nu^3} d\omega  \\
        N &= \int_0^{\omega_{max}} \frac{V}{2 \pi^2} \frac{\omega^2}{\nu^3} d\omega \\
        \omega_{max} &= \omega_D = \nu^3 \sqrt{\frac{6 \pi^2 N}{V}} = \frac{\nu}{a} \left(6 \pi^2\right)^{1/3} \\
        D(\omega) &= \frac{V}{2 \pi^2} \omega^2 \left(\frac{1}{\nu_l^3} + \frac{2}{\nu_t^3}\right) = \frac{3 V}{2 \pi^2} \frac{\omega^2}{\nu_D^3}
    \end{aligned}
\end{equation*}

\pagebreak
Für nicht-lineare Dispersion:
\begin{equation*}
    \begin{aligned}
        D(\omega)d \omega &= \frac{V}{2 \pi^2} q^2 \frac{dq}{d\omega} d\omega = \frac{V}{2 \pi^2} \frac{q^2}{v_g} d\omega
    \end{aligned}
\end{equation*}

\subsubsection*{Wärmekapazität – Debye Näherung}
\begin{equation*}
    \begin{aligned}
        U &= \int_0^{\omega_D} \hbar \omega D(\omega) \langle n(\omega, T) \rangle d\omega \\
        \langle n(\omega,T) \rangle &= \frac{1}{\exp(\hbar \omega / k_B T) -1} \\
        U(T) &= \frac{9N}{\omega_D^3} \int_0^{\omega_D} \frac{1}{\exp(\hbar \omega / k_B T) -1} d\omega \\
        \Theta &= \frac{\hbar \omega_D}{k_B} \qquad \text{Debye-Temp.} \\
        C_V &= \left(\frac{\partial U}{\partial T}\right)_V = 9Nk_B \left(\frac{T}{\Theta}\right)^3 \int_0^{x_D} \frac{x^4 e^x}{(e^x -1)^2} dx \\
        x &= \frac{\hbar \omega}{k_B T}; \quad x_D = \frac{\hbar \omega_D}{k_B T} = \frac{\Theta}{T}
    \end{aligned}
\end{equation*}

Für hohe Temperaturen: 
\begin{equation*}
    C_V = 3N k_B
\end{equation*}

Für tiefe Temperaturen:
\begin{equation*}
    C_V = \frac{12 \pi^4}{5} N k_B \left(\frac{T}{\Theta}\right)^3
\end{equation*}

Zahl der angeregten Phononen – Debye Näherung:
\begin{equation*}
    \begin{aligned}
        N_{ph} &= \int_0^{\omega_D} D(\omega) \langle n (\omega,T) \rangle d\omega \\ 
        &= \frac{3V}{2 \pi^2 \nu_D^3} \left(\frac{k_B T}{\hbar}\right)^3 \int_0^{x_D} \frac{x^2}{e^x -1}dx \\
        N_{ph} &\propto \begin{cases}
            T^3 & \, \text{für} T \ll \Theta \\
            T &  \, \text{für} T \gg \Theta
            \end{cases} \\
    \end{aligned}
\end{equation*}

\subsubsection*{Wärmekapazität – Einstein Näherung}
\begin{equation*}
    \begin{aligned}
        \langle n(\omega,T) \rangle &= \frac{1}{\exp(\hbar \omega / k_B T) -1} + \frac{1}{2}\\
        D(\omega) &= 3N \delta(\omega-\omega_E) \\
        C_V &= 3N k_B \left(\frac{\Theta_E}{T}\right)^2 \frac{\exp\left(\frac{\Theta_E}{T}\right)}{\left[\exp \left(\frac{\Theta_E}{T}\right)-1\right]^2} \\
        \Theta_E &= \frac{\hbar \omega_E}{k_B}
    \end{aligned}
\end{equation*}

\subsubsection*{Wärmeleitfähigkeit}
Fourrier-Gleichung für Wärmefluss:
\begin{equation*}
    j_Q = - \lambda \nabla T
\end{equation*}
kinetische Gastheorie (Phononen = ideales Gas):
\begin{equation*}
    \lambda = \frac{1}{3} C \nu l
\end{equation*}

\begin{itemize}
    \itemsep 0pt
    \item $C$ spez. Wärme
    \item $\nu$ mittlere Geschwindigkeit 
    \item $l$ mittlere freie Weglänge
\end{itemize}

\begin{equation*}
    \begin{aligned}
        \lambda &= \frac{1}{3} \sum_j \int_0^{\omega_{max}} c_j(\omega) \nu_j(\omega) l_j(\omega) d\omega \\
        c_j(\omega) &= \frac{dC_j}{d\omega}
    \end{aligned}
\end{equation*}

Mehrere unabhängigen Streumechanismen:
\begin{equation*}
    \frac{1}{l} = \frac{1}{l_A} + \frac{1}{l_B} + \dots
\end{equation*}