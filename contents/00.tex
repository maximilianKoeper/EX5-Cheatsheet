\section{Molekülanregungen}

% ###############################

\begin{equation*}
    \begin{aligned}
        E &= h \nu; \qquad \nu = \frac{c}{\lambda} \\
        E &= hc \cdot \frac{1}{\lambda} \\
        M &= \frac{M_1 M_2}{M_1 + M_2}
    \end{aligned}
\end{equation*}

% ###############################
\subsubsection*{2-atomiges Molekül als starrer Rotator}

\begin{equation*}
    \begin{aligned}
        E_{rot} &= \frac{1}{2} I \omega^2 = \frac{J^2}{2 I} \\
        I &= M_1R_1^2 + M_2R_2^2 = \left(\frac{M_1M_2}{M_1 + M_2}\right) R^2 \\
        J^2 &= J(J+1) \hbar^2 ; \qquad \abs{J} = I\omega \\
        E_{rot} &= \frac{J(J+1)\hbar^2}{2I} = \frac{J(J+1)\hbar^2}{2MR^2}\\
        \nu_{J \rightarrow J+1} &= \frac{\hbar}{2 \pi I} \left(J+1\right) \\
        &= \frac{1}{h} (E_{rot}(J+1) - E_{rot}(J)) \\
        \frac{N_J}{N_0} &= \left(2J+1\right) \exp \left(\frac{-E_J}{k_BT}\right)
    \end{aligned}
\end{equation*}

\subsubsection*{Verbindungsanregungen}

\begin{equation*}
    \begin{aligned}
        E_\nu &= \left(\nu + \frac{1}{2}\right) \hbar \omega \\
        \omega &= \sqrt{\frac{\omega}{M}}; \quad \Delta \nu &= \pm 1 \\
    \end{aligned}
\end{equation*}

\subsubsection*{kombinierte Spektren}

\begin{equation*}
    \begin{aligned}
        E_{\nu, J} &= E_\nu + E_J \\
        E_{\nu, J} &= \left(\nu + \frac{1}{2}\right) \hbar \omega + J(J+1) \frac{\hbar^2}{2I}
    \end{aligned}
\end{equation*}

\begin{itemize}
    \itemsep 0pt
    \item betrachte: $\nu = 0 \rightarrow \nu = 1$
    \item $\Delta J = -1$ P-Zweig $\omega_P = \omega - J \frac{\hbar}{I}$
    \item $\Delta J = 0$ Q-Zweig (meist verboten)
    \item $\Delta J = +1$ R-Zweig $\omega_R = \omega + (J+1)\frac{\hbar}{I}$
\end{itemize}

\subsubsection*{Auswahlregel (2-atomige Moleküle)}

\begin{equation*}
    \begin{aligned}
        M_{ij} &= \int \psi_i^* p \psi_j d\tau_{el} d\tau_N \\
        p &= -e \sum_i r_i + Z_1 eR_1 + Z_2 e R_2 = p_{el} + p_N \\
        \psi(r,R) &= \chi_N(R) + \varphi(r) \rightarrow \\
        M_{ij} &= \int \chi_i^* \left[\int \varphi_i^* p_{el} \varphi_j d\tau_{el}\right] \chi_j d\tau_{N} + \\
                &= \int \chi_i^* p_N \left[\int \varphi_i^* \varphi_j d\tau_{el}\right] \chi_j d\tau_{N}
    \end{aligned}
\end{equation*}

\begin{equation*}
    \begin{aligned}
        \Downarrow \quad &(|i\rangle = |j\rangle) \rightarrow (\varphi_i = \varphi_j) \\
        &\text{e-Orbitale ändern sich nicht} \\
        \rightarrow M_{ij} &= \int \chi_i^* p_N \chi_j d\tau_N \\
        \Downarrow \quad &(|i\rangle \neq |j\rangle) \\
        &\text{e wechselt Orbital} \\
        \rightarrow M_{ij} &= \int \chi_i^* \left[\int \varphi_i^* p_{el} \varphi_j d\tau_{el}\right] \chi_j d\tau_{N}
    \end{aligned}
\end{equation*}
$R_{min}$ verschiebt sich um $\Delta R$ \\
Kerne fangen an zu schwingen.