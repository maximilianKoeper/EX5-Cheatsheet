\section{Bindung im Festkörper}

% ###############################
\subsubsection*{Lennard-Jones Potential}
\begin{equation*}
    \begin{aligned}
        \varphi(r) &= \frac{\mathcal{A}}{r^{12}} \; \text{ oder } \varphi(r) = A' e^{-r/\rho}  \text{ Abstoßung} \\
        \varphi(r) &= - \frac{\mathcal{B}}{r^6} \text{ Van-der-Waals} \\
        \varphi(r) &= \frac{\mathcal{A}}{r^{12}} - \frac{\mathcal{B}}{r^6} \equiv 4 \epsilon \left[\left(\frac{\sigma}{r}\right)^{12}-\left(\frac{\sigma}{r}\right)^6\right]
    \end{aligned}
\end{equation*}

% ###############################
\subsubsection*{Bindungsenergie von Edelkristallen}
\begin{equation*}
    \begin{aligned}
        U_B &= \frac{1}{2} \sum_m \varphi_m = \frac{N}{2} \varphi_m \\
            &= 2N\epsilon \sum_{n \neq m} \left[\left(\frac{\sigma}{r_{mn}}\right)^{12}- \left(\frac{\sigma}{r_{mn}}\right)^6\right] \\
        r_{mn} &= p_{mn} R \; \text{, R: Abstand direkte Nachbarn}\\
            & p_{mn} \text{: z.B fcc-Kristall:} = 1, \sqrt{2}, 2, \dots \\
        0 &= \left. \frac{dU_B}{dR} \right|_{R_0} \Rightarrow R_0 = 1,0902 \sigma \\
        \rightarrow &U_B (R_0 ) = -8,61 N \epsilon
    \end{aligned}  
\end{equation*}

% ###############################
\subsubsection*{Ionenbindung}
\begin{equation*}
    \begin{aligned}
        \varphi_m &= \sum_{n \neq m} \left[\frac{\mathcal{A}}{r_{mn}^{12}} \pm \frac{e^2}{4 \pi \epsilon_0 r_{mn}}\right] \\
            &\approx z \frac{\mathcal{A}}{R^{12}} - \sum_{n \neq m} \frac{\pm e^2}{4 \pi \epsilon_0 p_{mn}R} = z  \frac{\mathcal{A}}{R^{12}} - \alpha \frac{e^2}{4 \pi \epsilon_0 R} \\
        \alpha & \equiv \sum_{n \neq m} \frac{\pm 1}{p_{mn}} \\
        U_B &= N \cdot \varphi_m \overset{Edelgaskr.}{=} - \frac{N \alpha e^2}{4 \pi \epsilon_0 R_0} \left(1- \frac{1}{12}\right)
    \end{aligned}
\end{equation*}

% ###############################
\subsubsection*{Kovalente Bindung ($H_2^+$)}
\begin{equation*}
    H = -\frac{\hbar^2}{2m} \Delta - \frac{e^2}{4 \pi \epsilon_0 r_a} - \frac{e^2}{4 \pi \epsilon_0 r_b} +\frac{e^2}{4 \pi \epsilon_0 R_{AB}}
\end{equation*}
LCAO-Methode:
\begin{equation*}
    \psi = c_1\varphi_a + c_2 \varphi_b
\end{equation*}
\begin{equation*}
    \begin{aligned}
        E &= \frac{\int \psi^* H \psi dV}{\int \psi^* \psi dV} = \frac{c_1^2H_{aa}+c_2^2H_{bb}+2c_1c_2H_{ab}}{c_1^2+c_2^2+2c_1c_2S} \\
        &H_{ij} = \int \psi_i^* H \psi_j dV , \; S= \int \psi_a^* \psi_b dV \\
        &E_{s;a} = \frac{H_{aa} \pm H_{ab}}{1 \pm S} + \frac{e^2}{4 \pi \epsilon_0 R_{AB}}
    \end{aligned}
\end{equation*}

% ###############################
\subsubsection*{Metallische Bindung}
$r_s$ definiert über: $\frac{V}{N} = \frac{4}{3} \pi r_s^3$
\begin{equation*}
    \begin{aligned}
        \frac{E_{coul}}{N} &= - \frac{e^2}{4 \pi \epsilon_0} \frac{9}{10r_s} \\
        \frac{E_{kin}}{N} &= \frac{3}{5} E_F = \frac{3}{5} \frac{\hbar^2}{2m_e} \left(\frac{9 \pi }{4}\right)^{\frac{2}{3}} \frac{1}{r_s^2}\\
        \frac{E_{aus}}{N} &= - \frac{3e^2}{16 \pi^2 \epsilon_0} \left(\frac{9 \pi}{4}\right)^{\frac{1}{3}} \frac{1}{r_s} \\
        \frac{E_B}{N} &= \left[-\frac{24.35}{(r_s/a_0)} + \frac{30.1}{(r_s/a_o)^2} - \frac{12.5}{(r_s/a_0)}\right] \frac{eV}{Atom} \\
        a_0 &= 0.529 \mathring{A}
    \end{aligned}
\end{equation*}

\subsubsection*{TODO: Pseudo-Potential}