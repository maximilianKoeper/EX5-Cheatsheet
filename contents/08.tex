\section{Elektronische Transporteigenschaften}

% ###############################

\subsubsection*{Effektive Masse}

\begin{equation*}
    \begin{aligned}
        \nu_n(k) &= \frac{1}{\hbar} \nabla_k E_n(k) = \frac{1}{\hbar} \frac{\partial E_n(k)}{\partial k} \\
        F &= \hbar \frac{dk}{dt} = -e \left[\mathcal{E}(r,t) + \nu (k) \times B(r,t)\right] \\
        \frac{d \nu_i}{dt} &= \sum_{j=1}^3 \left(\frac{1}{m^*}\right)_{ij} F_j \\
        \left(\frac{1}{m^*}\right)_{ij} &= \frac{1}{\hbar} \frac{\partial^2 E(k)}{\partial k_i \partial k_j}
    \end{aligned}
\end{equation*}

% ###############################

\subsubsection*{Ladungstransport}
\begin{equation*}
    \begin{aligned}
        j &= - \frac{e}{V} \sum_k \nu(k) = - \frac{e}{V} \int \rho_k \nu (k) f(E,T) d^3k \\
            &= \frac{-e}{4 \pi^3} \int \nu (k) f(E,T) d^3k \\
        T &\rightarrow 0: \\
        j &= \frac{-e}{4 \pi^3 \hbar} \int_{besetzt} \nabla_k E(k) d^3k
    \end{aligned}
\end{equation*}

\subsubsection*{Ladungstransport – Löcher}

\begin{equation*}
    \begin{aligned}
        j &= \frac{-e}{4 \pi^3} \int_{besetzt} \nu(k) d^3k \\
          &= \frac{-e}{4 \pi^3} \left[\int_{BZ} \nu(k) d^3 k - \int_{leer} \nu(k) d^3k\right] \\
          &= \frac{+e}{4 \pi^3} \int_{leer} \nu(k) d^3k
    \end{aligned}
\end{equation*}

Driftgeschwindigkeit:
\begin{equation*}
    \nu_d = (v-v_{th}) = -\frac{e \tau}{m} \mathcal{E} = -\mu \mathcal{E}
\end{equation*}

Beweglichkeit / Mobilität:
\subsubsection*{Sommerfeld-Theorie}

\begin{equation*}
    \begin{aligned}
        \sigma &= n e \mu = \frac{n e^2 \tau(E_F)}{m} \\
        j &= \sigma_{el} \epsilon
    \end{aligned}
\end{equation*}